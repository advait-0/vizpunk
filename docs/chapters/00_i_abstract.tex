%Two resources useful for abstract writing.
% Guidance of how to write an abstract/summary provided by Nature: https://cbs.umn.edu/sites/cbs.umn.edu/files/public/downloads/Annotated_Nature_abstract.pdf %https://writingcenter.gmu.edu/guides/writing-an-abstract
\chapter*{\center \Large  Abstract}
%%%%%%%%%%%%%%%%%%%%%%%%%%%%%%%%%%%%%%
% Replace all text with your text
%%%%%%%%%%%%%%%%%%%%%%%%%%%%%%%%%%%

This project develops a compact cost-effective development kit that integrates a function generator, power supply, and oscilloscope into a single portable device. Traditional lab equipment is often bulky, expensive, and lacks portability, limiting accessibility for individuals working on small-scale or field projects. By combining key tools into a single, affordable unit, we aim to enable budding electronics enthusiasts and hobbyists to carry out electronic tests and diagnostics easily.\\
The kit leverages a microcontroller to control the core functionalities, using ADC (Analog-to-Digital Converter) to capture signal inputs for the oscilloscope, and DAC (Digital-to-Analog Converter) to generate precise waveforms for the function generator. The development kit provides an affordable, portable, and user-friendly solution for electronic testing and diagnostics. It reduces the need for multiple bulky devices, making it easier for DIY enthusiasts, hobbyists, and professionals to conduct experiments and projects.

%%%
~\\[1cm]%REMOVE THIS
\noindent\textbf{Guidance on abstract writing:} An abstract is a summary of a report in a single paragraph up to a maximum of 250 words. An abstract should be self-contained, and it should not refer to sections, figures, tables, equations, or references. An abstract typically consists of sentences describing the following four parts: (1) introduction (background and purpose of the project), (2) methods, (3) results and analysis, and (4) conclusions. The distribution of these four parts of the abstract should reflect the relative proportion of these parts in the report itself. An abstract starts with a few sentences describing the project's general field, comprehensive background and context, the main purpose of the project; and the problem statement. A few sentences describe the methods, experiments, and implementation of the project. A few sentences describe the main results achieved and their significance. The final part of the abstract describes the conclusions and the implications of the results to the relevant field.


%%%%%%%%%%%%%%%%%%%%%%%%%%%%%%%%%%%%%%%%%%%%%%%%%%%%%%%%%%%%%%%%%%%%%%%%%s
~\\[1cm]
\noindent % Provide your key words
\textbf{Keywords:} a maximum of five keywords/keyphrase separated by commas

\vfill
\noindent
\textbf{Report's total word count: Following the abstract, the word count must be stated.} We expect at least 10,000 words in length and at most 15,000 words (starting from Chapter 1 and finishing at the end of the conclusions chapter, excluding references, appendices, abstract, text in figures, tables, listings, and captions), about 40 - 50 pages. \newline
\newline
\noindent
\textbf{Program code should be uploaded to gitlab, and the gitlab link should be included alongside the word count, following the abstract.} \newline
\newline
You must submit your dissertation report (preferred in a PDF file) via the “Turnitin assignment” in Blackboard Learn by the deadline. If a student has resits from the taught modules, the dissertation deadline will be extended for 3 weeks from the original dissertation deadline.

